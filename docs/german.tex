\documentclass[a4paper,11pt]{amsart}
\usepackage[ngerman,german,english]{babel}
\usepackage[utf8]{inputenc}
\usepackage{geometry}
\usepackage{booktabs}
\usepackage{graphicx}
\usepackage{listings}

\title{Automatisches Generieren schematischer Verkehrsnetzkarten als Problem der ganzzahligen Optimierung (MIP)}
\author{Julius Tens, Dirk Schumacher}
\email{tensjuli@math.hu-berlin.de}
% \date{1. August 2018}

\begin{document}
\maketitle

\section*{Problem}

\noindent Gegeben seien ein \textit{einfacher}, \textit{ungerichteter} Graph $G(V,E,L)$ bestehend aus Knoten $V$, Kanten $E$ und Linien $L$, sowie eine \textit{geradlinig planare} Einbettung $P$ dieses Graphen in die Ebene. Zu jedem Knoten $v_n$ sei also ein kartesisches Koordinatenpaar $(x_n | y_n) \in P$ bekannt (in der Regel über eine geeignete Projektion aus Geokoordinaten ermittelt), wobei es keine Schnittpunkte zwischen nicht benachbarten Kanten geben darf.
\\\\
Gesucht sind nun weitere \textit{geradlinige} Einbettungen $P'$ desselben Graphen $G(V,E,L)$ in die Ebene, die die folgenden Bedingungen erfüllen:
\bigskip

\begin{description}
\item[Oktilinearität] Alle Kanten müssen Strecken sein, die parallel oder senkrecht zur x-Achse oder zur Identität verlaufen.
\bigskip
\item[Bewahrung der Regionen] $P'$ enthält die gleichen Regionen (\textit{Faces}) wie $P$, d.h. die „Sortierung der Kanten“ an jedem Knoten muss gleich bleiben.
\bigskip
\item[Mindestlängen] Jede Kante muss eine Länge $l_e >= l_{min}$ haben.
\bigskip
\end{description}

\noindent Es wird insbesondere eine Einbettung $P'$ gesucht, die möglichst kurze Kantenlängen sowie wenige, bestenfalls stumpfwinklige „Knicke“ pro Linie hat.
\bigskip
\bigskip
\bigskip

% todo

\end{document}
